\documentclass[25pt,a0paper,portrait]{tikzposter}
\usepackage[utf8]{inputenc}
\usepackage{graphicx}
\usepackage{amsmath,amssymb}
\usepackage{blindtext}
\usepackage{lipsum}
\usepackage{graphicx}
\usepackage{algorithm}
\usepackage{algpseudocode}
\usepackage{caption}
\usepackage{hyperref}
\usepackage{adjustbox}

\usetheme{Rays}
\graphicspath{{Images/}}

\title{AI Methods for Post-Quantum Cryptography}
\author{
  Supervisor: Keqin Liu\\
  Members: Bo Wang, Yuan Cheng, Xingyue Fan, Yongrun Huang, Chang Liu, Jialun Luo, Yexi Ren
}
\institute{CODE: SURF-2025-0061}
\date{August 2025}

\makeatletter
  % 设置标题区占页面宽度的比例
  % 默认 \TP@titlewidth = \paperwidth
  \setlength\TP@titlewidth{0.8\paperwidth}   
  
  % 设置标题区高度(默认大约 9.5cm-10cm 左右,具体看字体大小)
  \setlength\TP@titleheight{5.5cm}             
  
  % 内部边距,如果需要文字离边框更远/更近,可调这个
  \TP@titleinnersep = 1.1em
  % \setlength\TP@titleinnersep{1.5cm}       
\makeatother
\begin{document}

\maketitle

\node[anchor=west,  xshift= -2.2cm] at (TP@title.west)
  {\includegraphics[width=0.15\paperwidth]{XJTLU1.png}};
\node[anchor=east,  xshift= 2cm] at (TP@title.east)
  {\includegraphics[width=0.13\paperwidth]{SURFwb.png}};



\block{Abstract}{
   This poster presents the background and a series of possible evolutionary methods of sieving algorithm-bgj3). We begin with reproducing the original random 
  filtering method, then introduce a deterministic Hybrid-Sieve to replace the randomness with the algebraic structure of 
  the dual lattice. Finally, we explore a powerful shift using Reinforcement Learning (RL), where an intelligent agent learns 
  to produce high quality center sieve vectors. 
}

\begin{columns}
  \column{0.55}
    \block{1. Introduction and Basic Knowledge}{
      % --- Start of the Poster Section Code ---

We begin by introducing three foundational elements that frame the methods and results that follow:

\medskip
\textbf{Lattice:}
A lattice in $\mathbb{R}^n$ is the discrete set of all integer linear combinations of basis vectors $b_1,\ldots,b_n$:
\[
\mathcal{L} =\Bigl\{\sum_{i=1}^{n} x_i b_i \;\Big|\; x _i\in \mathbb{Z} \Bigr\}.
\]
The basis is not unique; its density is captured by the determinant $\det(\mathcal{L} )=\lvert\det([b_1\,\cdots\,b_n])\rvert$.
Lattice problems form the foundation of many post-quantum schemes.

\medskip
\textbf{Shortest Vector Problem:}
Given a basis of $\mathcal{L}$, the task is to find a nonzero vector of minimum Euclidean norm:
\[
v^{\star} \in \operatorname*{arg\,min}_{v \in \mathcal{L}  \setminus \{\mathbf{0}\}} \|v\|_2 .
\]
SVP is believed to be hard in high dimensions and underlies the security of lattice-based cryptography (e.g., SIS/LWE, NTRU).
Related problems include the Closest Vector Problem (CVP) and Bounded Distance Decoding (BDD).

\medskip
\textbf{LLL Algorithm:}
The Lenstra–Lenstra–Lovász algorithm runs in polynomial time to produce a reduced basis with shorter, nearly orthogonal vectors.
It uses Gram–Schmidt coefficients $\mu_{i,j}$, size reduction ($|\mu_{i,j}|\le 1/2$), and the Lovász condition with $\delta\in(1/4,1)$.
A classical guarantee is $\|b_1\|\le 2^{\frac{(n-1)}{2}}\lambda_1(\mathcal{L})$, where $b_1$ is the first vector of the LLL-reduced basis. LLL is a standard pre-processing step for approximating SVP and for cryptanalysis (e.g., as a front-end to BKZ and sieving).

\medskip
These concepts underpin modern lattice algorithms: SVP sets the target, and LLL conditions the basis for practical
approximations. Building on these foundations, we study and refine sieving methods and sketch RL-based extensions.
}

\block{3. Algorithmic Enhancements}{

  We introduce two gradually deepening improvements to 
  \textsc{AllPairSearch}(bgj3):

  \begin{enumerate}
    \item \textbf{Deterministic Voronoi Multi-Level Sieve:}
      \begin{itemize}
        \item \textbf{Enumerated Dual Centers:}
          Precompute the top $N_{\rm enum}$ shortest vectors of the dual
          lattice $\mathcal{L}^\vee$ (via enumeration) as all sieve centers, replacing random picked center vectors.
        \item \textbf{Algebraic Pre‐filtering.}
          Exploit the algebraic structure of $\mathcal{L}^\vee$ (inner‐product bounds, norm relations)
          to perform a coarse sieve that discards vectors provably far from any optimal center with 
        \item \textbf{Voronoi Partitioning:}
          Split centers into disjoint sets $C_0,C_1,C_2$ (by $(B_0,B_1,B_2)$)
          and assign each vector to exactly one bucket via its closest center:\ $min \|\mathbf{v}\pm \mathbf{c}_i\|_2$
      \end{itemize}
      \textbf{Key Benefits:}
      \begin{itemize}
        \item \emph{Reproducible \& Deterministic.}
        \item \emph{Zero Overlap.}  Voronoi buckets are strictly disjoint. 
        \item \emph{Controlled Complexity.}
          Cost per level $O\bigl(|L|\cdot|C_i|\cdot n\bigr)$, lower than random caps. Friendly to the requirements of computing time and RAM.
      \end{itemize}

    \item \textbf{Reinforcement Learning–Based Center Selection:}
      \begin{itemize}
        \item \textbf{RL Environment:}
        State $s_t=[c_1^{(t)},\dots,c_n^{(t)}]\in\mathbb{Z}^n$;\\
        action $a_t\in\{c_i^+,c_i^-\}$;\\
        reward $R_t=\|\mathbf v_t\|^2-\|\mathbf v_{t+1}\|^2$.
        \item \textbf{Policy Network:}
          A multi‐layer perceptron $f_\theta$ outputs
          $\pi_\theta(a_t\mid s_t)$ over the action set.
        \item \textbf{Policy Optimization:}
          Use REINFORCE to maximize
          $J(\theta)=\mathbb{E}_{S\sim d}[v_\pi(S)]$,\\ updating
          $\theta\leftarrow\theta+\alpha\,\nabla_\theta\ln\pi_\theta(a_t|s_t)\,q_t$.
        \item \textbf{Application:}
          The trained agent proposes high‐quality sieve centers on the
          dual lattice.
      \end{itemize}
      \textbf{Key Benefits:}
      \begin{itemize}
        \item \emph{Adaptive Selection.} Learns to pick centers that drive
          faster convergence.
        \item \emph{Memory Efficient.} Avoids storing large random cap tables
          by focusing on promising lattice directions.
      \end{itemize}

  \end{enumerate}
}
    

  \column{0.45}
\block{2. Algorithms Re-implementation}{%
  % 整体字号
  \small
  % 取消多余段前/段后距离
  \setlength{\parskip}{0pt}%
  \setlength{\parsep}{0pt}%

  We adopted a combination of Python and C++ in Kaggle to reproduce the results of Prof.Ding's paper under the constraint of limited memory. The algorithms include the classical \textbf{sieving algorithm}, a \textbf{refined BGJ15}, and \textbf{bgj3} with a three-stage filtering scheme.
  Our training data comes from websites that generate random n-dimensional Lattice:
  https://www.latticechallenge.org/lwe_challenge/challenge.php 
  
  The following is the pseudocode of the original bgj3-algorithm:

  % 算法标题,下面用一个负距离把和上面正文的空隙顶回去
  \noindent\textbf{Algorithm\,1: AllPairSearch – bgj3 (Baseline)}\\[-2ex]
  % 上横线,宽度 0.9\linewidth,厚度 0.8pt,下面再顶回一点
  \noindent\rule{0.9\linewidth}{0.8pt}\\[-3ex]
  % 算法主体,只用 algorithmic
  \begin{algorithmic}[1]
    % 列表间距全部压小
    \setlength{\topsep}{0pt}%
    \setlength{\partopsep}{0pt}%
    \setlength{\itemsep}{.3ex}%
    \setlength{\parskip}{0pt}%
    \Require A list $L$ of $N_0$ lattice vectors, repetitions $(B_0,B_1,B_2)$, radius $(\alpha_0,\alpha_1,\alpha_2)$, goal norm $l$.
    \Ensure A list of reducing pairs in $L$.
    \State $\mathcal{N}\gets\emptyset$
    \For{$i=0,\dots,B_0-1$}
      \State Pick a random center $c_0\in S^{n-1}$
      \State $L_i\gets\{v\in L\mid v\text{ passes }F_{c_0,\alpha_0}\}$
      \For{$j=0,\dots,B_1/B_0-1$}
        \State Pick a random center $c_1\in S^{n-1}$
        \State $L_{ij}\gets\{v\in L_i\mid v\text{ passes }F_{c_1,\alpha_1}\}$
        \For{$k=0,\dots,B_2/B_1-1$}
          \State Pick a random center $c_2\in S^{n-1}$
          \State $L_{ijk}\gets\{v\in L_{ij}\mid v\text{ passes }F_{c_2,\alpha_2}\}$
          \State $\mathcal{N}\gets\mathcal{N}\cup\{(u,v)\in L_{ijk}^2\mid\|u\pm v\|<l\}$
        \EndFor
      \EndFor
    \EndFor
    \State \Return $\mathcal{N}$
  \end{algorithmic}\\[-2ex]
  % 下横线,和上面一样的宽度/高度
  \noindent\rule{0.9\linewidth}{0.8pt}
}


    \block{4. Key Results}{
      \lipsum[1-2]
      \lipsum[1][1-6]
      \begin{tikzfigure}
        \includegraphics[scale=0.8]{XJTLU1.png}
      \end{tikzfigure}
    }
    \block{5. Future Work}{
      \lipsum[1][1-5]
    }
\end{columns}

\block{References}{
    [1] Chinberg, T., Kalbach, A. LLLAlgorithmforLatticeBasisReduction. Available from: arXiv:2410.22196v2 [math.NT] 20 Nov 2024.

    [2] Williams, R.J. Simple statistical gradient-following algorithms for connectionist reinforcement learning. \textit{Mach Learn} 8, 229–256 (1992). https://doi.org/10.1007/BF00992696

    [3] Zhao, Z., Ding, J. and Yang, B.-Y. (2025). Sieving with Streaming Memory Access. \textit{IACR Transactions on Cryptographic Hardware and Embedded Systems}, 2025(2), 362-384. Available from: https://doi.org/10.46586/tches.v2025.i2.362-384 

}

\end{document}